\documentclass[a4paper,12pt]{article}
\usepackage[english]{babel}
\usepackage[utf8]{inputenc}

%
% For alternative styles, see the biblatex manual:
% http://mirrors.ctan.org/macros/latex/contrib/biblatex/doc/biblatex.pdf
%
% The 'verbose' family of styles produces full citations in footnotes, 
% with and a variety of options for ibidem abbreviations.
%
\usepackage{csquotes}
\usepackage[style=verbose-ibid,backend=bibtex]{biblatex}
\bibliography{sample}

\usepackage{lipsum} % for dummy text

\title{Text analysis of Trump’s tweets confirms he writes only the
(angrier) Android half}

\author{David Robinson}

\date{August 2016}

\begin{document}
\maketitle

\section{Introduction}
I don’t normally post about politics (I’m not particularly savvy about polling, which is where
data science has had the largest impact on politics). But this weekend I saw a hypothesis about
Donald Trump’s twitter account that simply begged to be investigated with data; see Figure 1.
When Trump wishes the Olympic team good luck, he’s tweeting from his iPhone. When he’s
insulting a rival, he’s usually tweeting from an Android. Is this an artifact showing which tweets
are Trump’s own and which are by some handler?
Others have explored Trump’s timeline and noticed this tends to hold up- and Trump himself
does indeed tweet from a Samsung Galaxy. But how could we examine it quantitatively? I’ve been
writing about text mining and sentiment analysis [1] recently, particularly during my development
of the tidytext R package with Julia Silge, and this is a great opportunity to apply it again.
My analysis, shown below, concludes that the Android and iPhone tweets are clearly from
different people, posting during different times of day and using hashtags, links, and retweets
in distinct ways. What’s more, we can see that the Android tweets are angrier and more
negative, while the iPhone tweets tend to be benign announcements and pictures. Overall I’d agree
with @tvaziri’s analysis: this lets us tell the difference between the campaign’s tweets (iPhone) and
Trump’s own (Android).

\section{The Dataset}
First we’ll retrieve the content of Donald Trump’s timeline using the userTimeline function in the
twitteR package:1

\printbibliography

\end{document}